% -*- TeX-master: "petalisp.tex"; TeX-engine: xetex; coding: utf-8 -*-
\section{Examples}
\label{sec:examples}

This section contains a collection of Petalisp programs and their
description to familiarize the reader with the previously defined
API.

\subsection{Linear Algebra}

The first batch of examples are from the domain of linear algebra, i.e. the
study of vector spaces and linear mappings between them. Elements and
operations on finite dimensional vector spaces are naturally represented as
vectors and matrices, which are both special cases of strided arrays.

The first example in figure \ref{fig:summation} shows two ways to compute
the sum of six integers, either using an application or a reduction. In
both cases, the functions α and β perform an implicit conversion of their
arguments from lisp objects to strided arrays.

\begin{figure}[h]
\resetlinenumber
\begin{code}
(α #'+ 1 2 3 4 5 6)
(β #'+ #(1 2 3 4 5 6))
\end{code}
\caption{Summation of integers in two different ways.}
\label{fig:summation}
\end{figure}

The dot product of two vectors $\mathbf{a}$ and $\mathbf{b}$ is defined as

\begin{align}
{ \mathbf {a} \cdot \mathbf {b} =\sum _{i=1}^{n}a_{i}b_{i}=a_{1}b_{1}+a_{2}b_{2}+\cdots +a_{n}b_{n}},
\end{align}

i.e. the sum of the products of the corresponding entries in both
vectors. The dot product is widely used, e.g. to compute physical
properties like the mechanical work or the magnetic flux. The corresponding
Petalisp program is shown in figure \ref{fig:dotproduct}.

\begin{figure}[h]
\resetlinenumber
\begin{code}
(β #'+ (α #'* a b))
\end{code}
\caption{The dot product of two vectors u and v.}
\label{fig:dotproduct}
\end{figure}

Norms are an essential tool for many areas of mathematics. One possible
norm for matrices is the row sum norm, which is defined as

\begin{align}
|A||_\infty = \max_{1 \le i \le m} \sum_{j=1}^n |a_{ij}|.
\end{align}

Matrices are the first example of multidimensional strided
arrays. According to definition \ref{def:reduction}, reductions apply to
the last dimension of a strided array, such that a reduction on a
$m \times n$ matrix yields a vector of length $m$ and the reduction of such
a vector yields a scalar.  The program in figure \ref{fig:rowsumnorm}
illustrates this convention, where the innermost reduction computes the sum
of the absolute values in each row, while the outer reduction forms the
maximum over the resulting row sums.

\begin{figure}[h]
\resetlinenumber
\begin{code}
(β #'max (β #'+ (α #'abs A)))
\end{code}
\caption{The row sum norm of a matrix A.}
\label{fig:rowsumnorm}
\end{figure}

The final linear algebra example is the matrix multiplication of two
matrices A and B. It is defined as

\begin{align}
C_{ij} = \sum_{p=1}^{n} A_{ip} B_{pj}.
\end{align}

The corresponding Petalisp program in figure \ref{fig:matmul} is the first
example to use the → function to reshape the matrices A and B. The
$m \times n$ matrix A is transformed to a $m \times 1 \times n$ array and
the $n \times k$ matrix B is transformed to a $1 \times k \times n$
array. Both arrays are then passed to the α function, which detects the
different shape of its arguments, broadcasts both to a common space of size
$m \times k \times n$ and multiplies them element-wise. In other words, $k$
reshaped instances of the matrix A and $m$ reshaped instances of the matrix
B are stacked next to each other before multiplying. In the end the last
dimension of the result of the α function (with size $n$) is summed up,
producing the result with shape $m \times k$.

\begin{figure}[h]
\resetlinenumber
\begin{code}
(β #'+
   (α #'*
      (→ A (τ (m n) (m 1 n)))
      (→ B (τ (n k) (1 k n))))))
\end{code}
\caption{The matrix multiplication of matrices A and B.}
\label{fig:matmul}
\end{figure}

Readers who think the matrix multiplication program in figure
\ref{fig:matmul} is more intricate than a classical version with three
nested loops are reminded that the code in figure \ref{fig:matmul} shows a
\emph{parallel} implementation of a matrix multiplication that may be used
to multiply matrices with a size of several terabytes on a supercomputer.
These four lines of Petalisp are roughly equivalent to parallel
implementations as in ScaLAPACK \cite{slug}, whose matrix multiplication
routine has several hundred lines of code.

\subsection{The Jacobi Method}

Most real world applications on parallel computers are concerned with
numerical algorithms. One such algorithm is the Jacobi method to solve
diagonally dominant systems of linear equations. Such systems arise
frequently in the discretization of partial differential equations
(PDEs). A simple and instructive example of such a PDE is the Laplace
equation with Dirichlet boundary conditions:

\begin{align}
- \Delta u &= 0 \qquad \text{on } \Omega \label{eq:laplace1} \\
 u &= C \qquad \text{on } \partial\Omega. \label{eq:laplace2}
\end{align}

\noindent Equation \ref{eq:laplace1} states that the divergence of the negative
gradient of a given quantity $u$ shall be zero at each point in the domain
$\Omega$. Equation \ref{eq:laplace2} provides the Dirichlet boundary conditions
for the problem, by assigning $u$ fixed values on each point of the domain
boundary $\partial\Omega$. In practice, the Laplace equation occurs e.g. when solving
heat diffusion problems. In this case, $u$ would be the temperature of the
medium. Heat flows from hot regions to colder ones, so the heat flux can be
modeled by the negative gradient of $u$. The conservation of energy implies
that the divergence of the heat flux is zero and leads exactly to the
Laplace equation.

For most domains and boundary values, there is no way to derive an analytic
solution to the Laplace equation. Instead the solution must be approximated
by numerical methods. A classical approach is to replace equation
\ref{eq:laplace1} by a linear system of equations that approximate the
Laplace operator at a finite set of grid points in the interior of $\Omega$. If
a two-dimensional domain is discretized by a rectangular grid with distance
$h$, a linear approximation of the Laplace equation at each point is given
by

\begin{align}
\label{eq:discretized-laplace}
- \Delta u_{x,y} \approx \frac{-4 u_{x,y} + u_{x+h,y} + u_{x-h,y} + u_{x,y+h} + u_{x,y-h}}{h^2} = 0.
\end{align}

\noindent The validity of this formula can be confirmed by Taylor expansion and
solving for the second derivatives. The Jacobi method approximates the
solution of such a system by starting with a random initial guess
$u_{0}$. Based on this initial guess, an improved guess $u_{k+1}$ is
obtained by solving the linear system point-wise, using the values of
$u_{k}$ for each neighbor:

\begin{align}
\label{eq:jacobi-update}
u_{k+1,x,y} = \frac{u_{k,x+h,y} + u_{k,x-h,y} + u_{k,x,y+h} + u_{k,x,y-h}}{4}
\end{align}

This update formula is applied in parallel to the interior point of the
discretized domain $\Omega$. The Dirichlet boundary values are never
modified. The Petalisp implementation of the Jacobi method for this
particular problem is shown in figure \ref{fig:Jacobi}. It starts in lines
2--4 by defining the interior of the given domain with the σ* macro. The
interior is then used in lines 10--13 to select exactly those points of $u$
that are in the interior \emph{after} the specified transformation. Finally
the fuse* function in line 7 determines the next value of $u_{k+1}$ by
overriding the interior of $u_{k}$ according to the update formula from
equation \ref{eq:jacobi-update}.

\begin{figure}[htb]
\resetlinenumber
\begin{code}
(defun jacobi-2d (u iterations)
 (let ((in
        (σ* u ((+ start 1) 1 (- end 1))
              ((+ start 1) 1 (- end 1)))))
   (loop repeat iterations do
     (setf u
       (fuse* u
        (α #'* 0.25
          (α #'+
             (→ u (τ (x y) (1+ x) y) in)
             (→ u (τ (x y) (1- x) y) in)
             (→ u (τ (x y) x (1+ y)) in)
             (→ u (τ (x y) x (1- y)) in))))))
   u))
\end{code}
\caption{The Jacobi Method to solve the equation $- \Delta u = 0$}
\label{fig:Jacobi}
\end{figure}

\subsection{Other examples}

There is a growing number of Petalisp examples on the Petalisp homepage
\footnote{\url{https://github.com/marcoheisig/Petalisp}}, ranging from
simple linear algebra examples to advanced techniques like multigrid
algorithms \cite{briggs2000}. The study of Petalisp examples is currently a
high priority, because we want to make sure that the language is prepared
for the parallel computing needs of the working computer scientist.
