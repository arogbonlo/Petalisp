% -*- TeX-master: "petalisp.tex"; TeX-engine: xetex; coding: utf-8 -*-
\documentclass[sigconf]{acmart}
\citestyle{acmauthoryear}

\usepackage{amsthm}
\theoremstyle{definition}
\newtheorem{define}{Definition}

% workaround to get Greek letters in typewriter font. Requires XeTeX
\usepackage{fontspec}
\setmonofont{DejaVu Sans Mono}[Scale=0.88]

% fancy boxes around source code examples
\usepackage{alltt}
\usepackage{calc}
\usepackage{fancybox}
\usepackage{lineno}
\setlength{\shadowsize}{1.5pt} \setlength{\fboxsep}{5pt}
\newenvironment{code}
{ \begin{Sbox} \begin{minipage}{\linewidth-26pt}
  \internallinenumbers \begin{alltt} }
{ \end{alltt} \end{minipage} \end{Sbox}
\makebox[\linewidth] { \mbox{\phantom{\tiny 99}}
\shadowbox{\TheSbox} } }

% Fancy display of function signatures
\newenvironment{function}
{\noindent
  \begin{minipage}{\linewidth}
    \setlength\tabcolsep{0pt}
    \begin{tabular}{p{0.86\linewidth}p{0.14\linewidth}}
      \rule{\linewidth}{1pt} & \rule{\linewidth}{1pt} \\
      \rule{0pt}{1.3em} }
{\rule{\linewidth}{1pt} & \rule{\linewidth}{1pt} \\ \end{tabular} \end{minipage} }

\def \N {\mathbb{N}}
\def \Z {\mathbb{Z}}
\def \Q {\mathbb{Q}}
\def \C {\mathbb{C}}

\setcopyright{rightsretained}

% \acmDOI{10.475/123_4}
% \acmISBN{123-4567-24-567/08/06}

\acmConference[ELS 2017]
{10th European Lisp Symposium}
{April 2017}
{VUB - Vrije Universiteit Brussel, Belgium}
\acmYear{2017}
\copyrightyear{2017}

\begin{document}
\title{Petalisp: A language for massively parallel computing}

\author{Marco Heisig}
\affiliation{%
  \institution{FAU Erlangen-N\"urnberg}
  \streetaddress{Cauerstra\ss e 11}
  \city{Erlangen}
  \postcode{91058}
  \country{Germany}
}
\email{marco.heisig@fau.de}

\begin{abstract}
  We present the design and our reference implementation of the parallel
  programming language Petalisp. We discovered that a simple data flow
  language is sufficient to express many practically relevant parallel
  algorithms. This programming model is a possible foundation for powerful
  parallel code generators.
\end{abstract}

\begin{CCSXML}
<ccs2012>
<concept>
<concept_id>10003752.10003753.10003761.10003762</concept_id>
<concept_desc>Theory of computation~Parallel computing models</concept_desc>
<concept_significance>500</concept_significance>
</concept>
<concept>
<concept_id>10011007.10011006.10011008.10011009.10010175</concept_id>
<concept_desc>Software and its engineering~Parallel programming languages</concept_desc>
<concept_significance>500</concept_significance>
</concept>
<concept>
<concept_id>10011007.10011006.10011008.10011009.10010177</concept_id>
<concept_desc>Software and its engineering~Distributed programming languages</concept_desc>
<concept_significance>500</concept_significance>
</concept>
<concept>
<concept_id>10011007.10011006.10011008.10011009.10011012</concept_id>
<concept_desc>Software and its engineering~Functional languages</concept_desc>
<concept_significance>300</concept_significance>
</concept>
<concept>
<concept_id>10011007.10011006.10011008.10011009.10011016</concept_id>
<concept_desc>Software and its engineering~Data flow languages</concept_desc>
<concept_significance>300</concept_significance>
</concept>
<concept>
<concept_id>10011007.10011006.10011041.10011044</concept_id>
<concept_desc>Software and its engineering~Just-in-time compilers</concept_desc>
<concept_significance>300</concept_significance>
</concept>
<concept>
<concept_id>10011007.10011006.10011041.10011048</concept_id>
<concept_desc>Software and its engineering~Runtime environments</concept_desc>
<concept_significance>100</concept_significance>
</concept>
</ccs2012>
\end{CCSXML}

\ccsdesc[500]{Theory of computation~Parallel computing models}
\ccsdesc[500]{Software and its engineering~Parallel programming languages}
\ccsdesc[500]{Software and its engineering~Distributed programming languages}
\ccsdesc[300]{Software and its engineering~Functional languages}
\ccsdesc[300]{Software and its engineering~Data flow languages}
\ccsdesc[300]{Software and its engineering~Just-in-time compilers}
\ccsdesc[100]{Software and its engineering~Runtime environments}

\keywords{High-Performance Computing, Common Lisp}

\maketitle

% -*- TeX-master: "petalisp.tex" -*-
\section{Introduction}
\label{sec:introduction}

The automatic parallelization of computer programs is the holy grail of
parallel computing. The search for this holy grail has spawned many
programming languages, such as High Performance Fortran \cite{HPF}, SISAL
\cite{SISAL}, Fortress \cite{Fortress}, SequenceL \cite{SequenceL}, CM Lisp
\cite{CM-Lisp}, Chapel \cite{Chapel}, Unified Parallel C \cite{UPC} and X10
\cite{X10}. Each of these languages offers a notation for parallel
algorithms and relieves the programmer form the tedious and error-prone
task of communicating data between processes. Yet the majority of parallel
programs in industry and scientific computing are written using classical
general-purpose programming languages. This is not surprising, since no
parallelizing compiler can match or outperform a human expert programmer.

But why is that so? Why are parallelizing compilers consistently worse than
human experts? The key difference is not so much the repertoire of known
optimization techniques, but the understanding of the given program. When a
programmer studies a program, she focuses on the high-level algorithm and
data structures. While doing so, she identifies and skips many irrelevant
sections. She might even find some bugs in the original code. This
illustrates that humans have knowledge far beyond the source code.

In contrast, a machine cannot make assumptions about the meaning of a
program. Without any noteworthy intelligence, it is forced to take the
whole program literally. Consequently, the machine has usually zero
knowledge of the high-level behavior of the program. It only sees a jungle
of data and control flow dependencies. Each of these dependencies can
prevent the whole program from parallelization, even if it belongs only to
a superfluous print statement. In other words, it is impossible to develop
a compiler that reliably parallelizes real world programs. The search for
the holy grail is indeed futile.

This does not mean that computers are bad at generating parallel
programs. The source code of a general purpose program is just a horrible
way to convey meaning to such a code generator. Instead we propose a
specialized programming language to express parallel algorithms at a higher
level. This language must be simple enough that every possible combination
of language primitives is intelligible, yet expressive enough to implement
real world algorithms.

In this paper, we describe the design and implementation of the parallel
programming language Petalisp. In section \ref{sec:examples} we show how it
can be applied to several important parallel algorithms.

A peculiarity of Petalisp is its tight integration into the Common Lisp
programming language. Common Lisp programs are used to create Petalisp
programs, and Petalisp operators take ordinary Common Lisp functions as
arguments\footnote{albeit undefined behavior occurs if these functions have
  side effects}. This approach has two advantages. The first one is that
Petalisp itself is a purely functional language, with all associated
benefits for analysis and optimization. The second one is that Petalisp has
access to every feature and library of Common Lisp. Effectively, Petalisp
is an extension of Common Lisp for massively parallel computing.

% -*- TeX-master: "petalisp.tex"; TeX-engine: xetex; coding: utf-8 -*-
\section{The Design of Petalisp}
\label{sec:design}

If the design goal of Petalisp had to be summarized in one sentence, it
would be:

\begin{quotation}
  Every Petalisp program can be executed efficiently on a parallel
  computer
\end{quotation}

\noindent A consequence of this goal is that other language qualities, such
as expressiveness, are sacrificed where necessary. The design process
thereby turns into a search for the minimal set of features that are
sufficient to denote the majority parallel algorithms. The result of this
minimization problem is a language with only a single data structure and
four primitive operators, which are described hereafter.

\subsection{Strided Arrays}
\label{sec:strided-arrays}

Computation is hardly meaningful without structured data. In classical
Lisps, this role is filled by the \texttt{cons} function, from which all
other data structures can be derived. However, this approach produces data
structures that are far too heterogeneous for any automatic
parallelization. Instead Petalisp uses a data structure that has far more
data regularity: strided arrays.

The viability of array oriented programming has been demonstrated by the
programming language APL \cite{APL}. Strided arrays are an extension of
classical arrays, where the valid indices in each dimension are denoted by
three integers: The smallest admissible index, the step size and the
highest admissible index. More precisely, strided arrays can be defined as:

\begin{define}[strided array]
  A strided array in $n$ dimensions is a function from elements of the
  cartesian product of $n$ ranges to a set of Common Lisp objects.
\end{define}

\begin{define}[range]
  A range with the lower bound $x_L$, the step size $s$ and the upper bound
  $x_U$, with $x_L,\, s,\, x_U \in \Z$, is the set of integers
  $$\{\, x \in \Z \;|\; x_{L} \le x \le x_{U} \;\wedge\; (\exists k \in \Z)\,[ x = x_{L} + k s]\,\}$$
\end{define}

The domain of an array is henceforth called its \emph{index space}.  An
index space is denoted by an S-expression, starting with the Greek letter σ
\,(for \textbf{s}pace or \textbf{s}hape), followed by lists of integers
denoting the lower bound, step size and upper bound of each range. If the
step size is one, it can be omitted. Figure \ref{fig:sigma-examples}
illustrates this convention.

\begin{figure}[h]
  \begin{tabular}{ll}
    \texttt{(σ)}                         & \hspace{-1em} the zero-dimensional space \\
    \texttt{(σ (0 1 8) (0 1 8))}         & \hspace{-1em} index space of a $9 \times 9$ array \\
    \texttt{(σ (0 8) (0 8))}             & \hspace{-1em} ditto \\
    \texttt{(σ (10 2 98))}               & \hspace{-1em} all even two-digit numbers \\
    \texttt{(σ (1 2 3) (1 2 3) (1 2 3))} & \hspace{-1em} corners of a $3 \times 3 \times 3$ cube \\
  \end{tabular}
  \caption{A notation for strided arrays.}
  \label{fig:sigma-examples}
\end{figure}

Strided arrays have several advantages over classical arrays. The lowest
and highest index can be chosen arbitrarily, including from the set of
negative integers \footnote{Coincidentally avoiding any debate over a
  canonical lowest array index.} --- a flexibility that is later used to
define translations of arrays and subsequently to define an unambiguous way
to stack multiple strided arrays next to each other.  The step size
parameter allows to model data that is not contiguous, yet has some level
of regularity.

The original motivation to support array strides stems from the observation
that many parallel algorithms have nearest neighbor data dependencies. To
parallelize them regardless, the domain must be partitioned into multiple
independent sets. This process is called coloring. A simple example of such
a coloring strategy is a chessboard, which ensures that the direct
neighbors of any white field are black, and vice versa. All elements with
the same color can be grouped into a small number of strided arrays. In the
case of the chessboard, two strided arrays are sufficient to describe all
tiles of the same color.

\subsection{Core Operations}

Before the four core operations of Petalisp are discussed, it is
instructive to discuss those features that are not present in Petalisp: The
language lacks any form of control flow --- no conditionals, no function
calls, no jumps and no mechanism for exception handling. Even more, there
is no mechanism for destructive assignment. As a result, a Petalisp program
is nothing more than a pure data flow graph.

The most prominent Petalisp operator is the distributed
\emph{application}. It describes embarrassingly parallel problems, where a
single function is applied to every element of one or more strided arrays,
much like the Lisp function \texttt{map}. A secondary source of parallelism
is introduced by the distributed \emph{reduction} operator, which is
similar to the Lisp function \texttt{reduce}, but where the order of
reduction is unspecified.

\begin{define}[application]
  Let $f$ be a referentially transparent\footnote{A function is
    referentially transparent if it has negligible side effects and same
    arguments always yield the same value.} Common Lisp function that
  accepts $n$ arguments, and let $a_{1}, \ldots, a_{n}$ be strided arrays
  with index space $\Omega$. Then the application of $f$ to
  $a_{1}, \ldots, a_{n}$ is a strided array that maps each index $k$ of
  $\Omega$ to $f( a_{1}(k), \ldots, a_{n}(k) )$.
\end{define}

\begin{define}[reduction]
  \label{def:reduction}
  Let $f$ be a referentially transparent Common Lisp function that accepts
  two arguments, and let $a$ be strided array of dimension $n$, i.e. a
  mapping from each element of the cartesian product of the ranges
  $R_{1}, \ldots, R_{n}$ to some values. Then the reduction of $a$ by $f$
  is a Petalisp data structure of dimension $n-1$ that maps each element
  $k$ of $R_{1} \,\times\, \ldots \, \times \, R_{n-1}$ to the combination of the
  elements $\{ a(i) \,|\, i \in k \,\times\, \Omega_{n} \}$ by $f$ in some
  arbitrary order.
\end{define}

Parallel application and reduction are the only two operators that actually
evaluate Common Lisp functions to compute new values. The remaining two
operators are the \emph{fusion} of several strided arrays, and
\emph{reference} operator, which allow to combine, destructure and reshape
existing strided arrays.

\begin{define}[fusion]
  \label{def:fusion}
  Let $a_{1}, \ldots, a_{n}$ be strided arrays with equal dimension, each
  mapping from an index space $\Omega_{k}$ to a set of values.
  Furthermore, let the sets $\Omega_{1}, \ldots, \Omega_{n}$ be pairwise
  disjoint, and let $\Omega_f = \bigcup_{k=1}^{n} \Omega_{k}$ be again a
  valid index space. Then the fusion of $a_{1},\ldots, a_{n}$ is a strided
  array that maps each index $i \in \Omega_f$ to the value of $i$ of the
  unique strided array $a_{k}$ whose index space contains $i$.
\end{define}

\begin{define}[reference]
  \label{def:reference}
  Let $a$ be a strided array with domain $\Omega_a$, let $\Omega_b$ be a
  strided array index space and let $t$ be a transformation from $\Omega_b$
  to $\Omega_a$. Then the reference of $a$ by $\Omega_b$ and $t$ is a
  strided array that maps each $i \in \Omega_b$ to $a(t(i))$.
\end{define}

What has not been specified so far is the space of permissible
transformations in definition \ref{def:reference}, i.e. the possible ways
to reshape an array. This choice is crucial. Limiting the space of
transformations extremely, e.g. only to the identity function, would render
the language more or less useless. A too liberal policy would increase the
complexity of the language to unmanageable levels, conflicting with our
design goal of providing reliable optimization. The next section describes
the compromise that was finally chosen.

\subsection{Permissible Transformations}
\label{sec:transformations}

Study of real world applications has determined, that the following
elementary transformations on strided arrays are particularly useful:

\begin{itemize}
\item \textbf{translation} of indices by a constant, e.g.
  \begin{flushleft}
    $\texttt{(σ (0 9))} \xrightarrow{+\,10} \texttt{(σ (10 19))}$
  \end{flushleft}
\item \textbf{scaling} of indices with a constant, e.g.
  \begin{flushleft}
    $\texttt{(σ (0 4))} \xrightarrow{\times\, 11} \texttt{(σ (0 11 44))}$
  \end{flushleft}
\item \textbf{permuting} the indices of a multidimensional array, e.g.
  \begin{flushleft}
    $\texttt{(σ (-4 4) (2 9))}
    \xrightarrow{1^{\text{st}} \,\leftrightarrow\, 2^{\text{nd}}}
    \texttt{(σ (2 9) (-4 4))}$
  \end{flushleft}
\item \textbf{shrinking} the dimension by dropping some indices, e.g.
  \begin{flushleft}
    $\texttt{(σ (2 2) (3 9))}
    \xrightarrow{\text{drop}\, 1^{\text{st}}}
    \texttt{(σ (3 9))}$
  \end{flushleft}
\item \textbf{increasing} the dimension by adding some indices,e.g.
  \begin{flushleft}
    $\texttt{(σ)}
    \xrightarrow{\text{add}\, 4,\, \text{add}\, 9}
    \texttt{(σ (4 4) (9 9))}$
  \end{flushleft}
\item \textbf{collapsing} one dimension to a single element, e.g.
  \begin{flushleft}
    $\texttt{(σ (2 19) (4 12)}
    \xrightarrow{\text{collapse}\, 1^{\text{st}}\, \text{to}\, 0}
    \texttt{(σ (0 0) (4 12))}$
  \end{flushleft}
\end{itemize}

These six elementary transformations and all possible combinations thereof
form the set of permissible transformations in Petalisp. Understanding and
implementing this set of transformation in Common Lisp is a central
component of this programming model. Transformations are treated as
first-class citizens in Petalisp and have their own notation, which is
similar to the notation of ordinary lambda functions, but starting with the
letter τ and with an implicit values form wrapped around the body, as seen
in figure \ref{fig:transformation-examples}

\begin{figure}[htb]
  \begin{tabular}{ll}
    \texttt{(τ ())} & \hspace*{-1em} the transformation from \texttt{(σ)} to \texttt{(σ)} \\
    \texttt{(τ (i) (+ i 2))} & \hspace*{-1em} shift all indices of a $1$D array by $2$ \\
    \texttt{(τ (m n) m (* n 1/2))} & \hspace*{-1em} scale the second dimension by $1/2$ \\
    \texttt{(τ (m n) n m)} & \hspace*{-1em} invert a matrix \\
    \texttt{(τ (5 a) a)} & \hspace*{-1em} drop the first index if it is 5 \\
    \texttt{(τ (a) a 9)} & \hspace*{-1em} add one dimension with index 9 \\
  \end{tabular}
  \caption{A notation for transformations.}
  \label{fig:transformation-examples}
\end{figure}

\subsection{The Petalisp API}

The core operators of Petalisp --- application, reduction, fusion and
reference --- are simple and orthogonal. They are designed to be easy to
reason about, but not to make it pleasant to write programs with them. This
concern is addressed by the Petalisp API, a set of powerful, smart
functions that expand into one or more invocations of Petalisp core
operators.

\begin{function}
  \texttt{ \textbf{σ} \&rest range-specifications} & \textsl{macro} \\
\end{function}

The macro σ parses the notation for strided arrays from section
\ref{sec:strided-arrays}. Each range specification must be a list of forms,
which are evaluated from left to right in the current environment to
produce the parameters of this range. Returns an object of type
\texttt{strided-array-index-space}.

\begin{function}
  \texttt{ \textbf{σ*} space \&rest range-specifications} & \textsl{macro} \\
\end{function}

Space must be a strided array index space, whose dimension matches the
number of range specifications. Each range specification is parsed as in
the macro σ, but with the variables start, step and end bound to the
respective values of the corresponding range in \texttt{space}. This macro
allows to describe a strided array index space relative to another one,
e.g. stripping the boundary of an existing space. Returns an object of type
\texttt{strided-array-index-space}.

\begin{function}
  \texttt{ \textbf{τ} arguments \&body body} & \textsl{macro} \\
\end{function}

Arguments must be a list of symbols, body must be a list of forms, as shown
in figure \ref{fig:transformation-examples}. The body forms are evaluated
multiple times, with symbols bound to some integers, to determine the
properties of the transformation. Only transformations as specified in
section \ref{sec:transformations} are permitted, an attempt is made to
signal an error in case of invalid transformations. Returns an object of
type \texttt{transformation}.

\begin{function}
  \texttt{ \textbf{α} function \&rest arguments} & \textsl{function} \\
\end{function}

Apply the application operator to the given \texttt{function} and the
\texttt{arguments}. If any of the latter is not a strided array, but a Lisp
scalar or array, it is suitably converted. If, after conversion, the
strided arrays have a different shape, an attempt is made to broadcast them
to a common shape using the reference operator. These references must not
permute, scale or translate their arguments. An error is signaled, when
there is no way to broadcast all strided arrays to a common shape. Returns
an object of type \texttt{strided-array}.

\begin{function}
  \texttt{ \textbf{β} function argument} & \textsl{function} \\
\end{function}

Apply the reduction operator to the given \texttt{function} and
\texttt{argument}. If the latter is a Lisp array, it is converted to a
strided array. Returns an object of type \texttt{strided-array}.

\begin{function}
  \texttt{ \textbf{→} strided-array \&rest modifiers} & \textsl{function} \\
\end{function}

The function → is the generalized reference and broadcast operator. If
\texttt{strided-array} is a Lisp array or scalar, it is suitably converted.
Afterwards, the \texttt{modifiers} are processed from left to right, using
the result of each modification as argument for the next one. Each modifier
must either be a strided array index space or an invertible
transformation. If the current modifier is an invertible transformation,
emit a reference node to the resulting index space and the inverse of this
transformation\footnote{The inverse of each transformation is used to
  accommodate for the different perspectives of programmers and data flow
  graphs.  Programmers prefer to say ``shift those elements to the right'',
  while the natural corresponding data flow node is ``these are the values
  of my predecessor, coming from the left.''}. If the current modifier is a
strided array index space and a subspace of the index space of the
argument, emit a reference that selects only this subspace. Otherwise
attempt to broadcast the argument to the given space with a suitable
reference. Returns an object of type \texttt{strided-array}.

\begin{function}
  \texttt{ \textbf{fuse} \&rest arguments} & \textsl{function} \\
\end{function}

All \texttt{arguments} are converted to strided arrays when necessary. The
resulting arrays must must not overlap and are passed to the fusion
operator. Returns an object of type \texttt{strided-array}.

\begin{function}
  \texttt{ \textbf{fuse*} \&rest arguments} & \textsl{function} \\
\end{function}

The fuse* function is similar to the fuse function, but the arguments may
overlap. For indices where some of the arguments overlap, the value of the
rightmost array is chosen. Returns an object of type
\texttt{strided-array}.

% -*- TeX-master: "petalisp.tex" -*-
\section{Examples}
\label{sec:examples}
% -*- TeX-master: "petalisp.tex"; TeX-engine: xetex; coding: utf-8 -*-
\section{Implementation}
\label{sec:implementation}

The Petalisp language is a pragmatic approach to parallel programming. Its
goal is not primarily to advance the theoretical understanding of
parallelism, but to derive a practical solution for modern scientific
computing. The development and maintenance of a high-quality Petalisp
implementation is a crucial step towards this goal.

Throughout the implementation, we faced two exciting challenges. On the one
hand, Petalisp must be fast enough to be competitive with existing software
in High-Performance Computing. On the other hand, since Petalisp is
university project, it must be simple enough that new developers can read
and understand the full source code within a few days.

\subsection{Evaluation of Petalisp Programs}

Petalisp programs themselves are not meaningful. There is (deliberately) no
way to access the contents of a strided array. The only way to do so is to
convert the result of a Petalisp program back to a Lisp array or scalar. An
advantage of this strategy is that the execution of the whole Petalisp
program can be delayed until such a conversion is requested. The result is
unprecedented potential for optimization, albeit with the price that all
compilation has to be performed just-in-time.

For some applications, it is desirable to compute multiple strided arrays,
but convert only some of them to Lisp objects, e.g. to check some control
parameters and pass the remaining arguments to another Petalisp program. To
accommodate this case, the evaluation of Petalisp programs is split into
two functions: \emph{compute} and \emph{petalisp->lisp}:

\begin{function}
  \texttt{ \textbf{compute} \&rest arguments} & \textsl{function} \\
\end{function}

Force the evaluation of all \texttt{arguments}. Returns as many values as
there have been arguments. All result values are of type
\texttt{strided-array}.

\begin{function}
  \texttt{ \textbf{petalisp->lisp} argument} & \textsl{function} \\
\end{function}

If \texttt{argument} has not been evaluated yet, do so. Afterwards, if it
is a strided array of dimension zero, convert it to a Lisp scalar. If it is
a strided array of higher dimension, translate it to an origin of zero,
scale it to a step size of one and convert it to a Lisp array.

\subsection{Data Flow Graph Optimization}

Each strided array is also an instance of one of the classes
\emph{constant}, \emph{application}, \emph{reduction}, \emph{fusion} or
\emph{reference}. All instances except constants track their predecessors
in the data flow graph. Therefore and because of lazy evaluation, a strided
array is more a recipe than a data structure. Figure \ref{fig:matmul-graph}
shows the data flow graph for an invocation of the matrix multiplication
program in figure \ref{fig:matmul} with a $1\times 2$ matrix A and a
$2 \times 5$ matrix B.

\begin{figure}[htb]
\begin{center}
  \begin{tikzpicture}
    [petanode/.style={
      rectangle,
      rounded corners = 3mm,
      minimum size=12mm,
      thick,draw=black,
      font=\ttfamily},
    arrow/.style={<-,shorten <=1pt,thick},
    align=center]
    \node (lconst)      [petanode] at (1,6) {constant\\(σ (1 3) (1 2))};
    \node (rconst)      [petanode] at (5,6) {constant\\(σ (1 2) (1 5))};
    \node (lrepetition) [petanode] at (1,4) {reference\\(τ (A B C) A C)\\(σ (1 3) (1 5) (1 2))}
     edge [arrow] (lconst);
    \node (rrepetition) [petanode] at (5,4) {reference\\(τ (A B C) C B)\\(σ (1 3) (1 5) (1 2))}
     edge [arrow] (rconst);
    \node (application) [petanode] at (3,2) {application (*)\\(σ (1 3) (1 5) (1 2))}
     edge [arrow] (lrepetition)
     edge [arrow] (rrepetition);
    \node (reduction)   [petanode] at (3,0) {reduction (+)\\(σ (1 3) (1 5))}
     edge [arrow] (application);
\end{tikzpicture}
\end{center}
\caption{The data flow graph of a matrix multiplication}
\label{fig:matmul-graph}
\end{figure}

There is a lot of potential for optimization during graph
creation. Multiple consecutive references can be combined to a single one
by forming the functional composition of their transformations. Operations
on small, constant arrays can be evaluated immediately, producing a new
constant. References to a fusion can directly reference the relevant inputs
of that fusion. These transformations are the first place where the
simplicity of the core language pays off. It is sufficient to consider all
interactions between four different operators.

To exploit the aforementioned simplifications as soon as possible, they are
performed directly within the core operators. Each core operator, e.g.
\emph{application} is a generic function that constructs a new data flow
graph node by default. Specialized functions override this behavior to
simplify or avoid this node construction where possible. A result of these
early transformations is that the data flow graph of a Petalisp program is
often much shorter than its generating source code, e.g. an invocation of
the code in figure \ref{fig:references} is completely optimized away.

\begin{figure}[htb]
\resetlinenumber
\begin{code}
(defun useless-references (x)
  (setf x (-> x (τ (x) (+  1 x))))
  (setf x (-> x (τ (x) (+  9 x))))
  (setf x (-> x (τ (x) (- 10 x)))))
\end{code}
\caption{Three consecutive translating references.}
\label{fig:references}
\end{figure}

\subsection{Code Generation}

Once the data flow graphs have been assembled and simplified, they are
evaluated. This evaluation consists of the following steps:

\begin{enumerate}
\item \textbf{grouping into kernels}\, As many nodes as possible are
  grouped to a single kernel. Kernel boundaries occur e.g. when one value
  is required from multiple sources. The matrix multiplication graph in
  figure \ref{fig:matmul-graph} would be grouped into a single kernel.

\item \textbf{kernel analysis}\, Each kernel is analyzed to approximate its
  computational cost and some auxiliary metrics.

\item \textbf{scheduling and allocation}\, The previously gathered
  information is used to derive a reasonable scheduling strategy and to map
  the value of each kernel to a memory location\footnote{This phase is similar to
  register allocation in classical compilers.}.
\item \textbf{execution}\, Communications and computations are performed
  according to the schedule. During communication, data is exchanged to
  ensure that all necessary information for a kernel invocation is locally
  available. During computation, a kernel is used as a blueprint to
  generate and compile specialized source code for this particular
  problem. Afterwards, this code is applied to the given input and output
  memory locations to compute the next value.
\end{enumerate}

The code generator is still under development, but nonetheless fully
functional. While it seems expensive to compile each kernel at runtime,
this cost is greatly diminished by caching all compiled kernels for future
use. Except for some pathological programs, compilation quickly becomes as
cheap as a single hash table lookup.

% -*- TeX-master: "petalisp.tex" -*-
\section{Conclusions}
\label{sec:conclusions}


\section{Acknowledgments}

\bibliographystyle{ACM-Reference-Format}
\bibliography{petalisp}

\end{document}
